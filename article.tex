\documentclass[journal]{IEEEtran}
\usepackage[pdftex]{graphicx}
\usepackage[table]{xcolor}
\usepackage{parskip}
\usepackage{float}
\usepackage{tikz}
\usetikzlibrary{shapes,arrows}

\newcommand{\inquote}[1]{\textit{``#1''}}
\newcommand{\cquote}[1]{\begin{center}
						\inquote{#1}
					\end{center}	
					}														
\newcommand{\cquestion}[1]{\begin{center}
							\textit{#1?}
						\end{center}
					}
					
\begin{document}
\title{\textbf{A Serious Game for Aiding the Screening of Dyslexia in Children and Young Adults}}
\author{Rose Tucker}

\maketitle

\begin{abstract}

The standard screening tests for detecting dyslexia in adolescents typically involve a number of written, spoken, and visual tests to be carried out by a specialist dyslexic teacher or psychologist.
This test lasts around half an hour and can cost the individual in excess of \textsterling300. 
This paper presents a more accessible and engaging method of screening individuals
 for dyslexia, through the use of a serious game. 
The game was tested on 42 participants, 7 diagnosed with dyslexia, and 35 controls which were seen to be in no way dyslexic. 
Testing showed that two gameplay performance metrics identify significant differences between the two groups, and that the game can identify diagnosed dyslexic subjects from those without dyslexia with an accuracy of 95.2\%, whilst providing an engaging experience for all users.
\end{abstract}

\section{Introduction}
\label{sec:intro}

\IEEEPARstart{T}{argeting} adolescents, this work examines whether the conventional screening
test for dyslexia could be replaced with a serious game. The game aims to use gameplay performance metrics of a player to predict wether they are likely to have dyslexia. The game also aims to screen individuals without explicitly highlighting the literacy and phonological deficits often associated with dyslexia, with the hope of 
producing an engaging experience for those with and without dyslexia. 
If the game is successful, it should vastly reduce the cost, personnel, and time taken to 
identify dyslexia through not requiring a specialist dyslexia teacher, or psychologist, to 
conduct a screening test \cite{bda, dast}. Hopefully the accessibility of the game, to 
teachers, parents, and students themselves, will result in more
young people being tested and receiving the specialist help they require.

The following examines current literature surrounding both dyslexia and games, to determine the breadth of current knowledge in both areas and show how combining the two fields could change what is state-of-the-art. This knowledge is then used to create a digital game built to screen for dyslexia. The created game is then evaluated through a user study.

\section{Dyslexia}
\label{sec:dyslexia}

Dyslexia is a specific learning disability which affects around 8-10\% of the UK population \cite{Nhs,bda}. It is often considered a continuum, with no distinct cut off and varying severity. Though there are disagreements as to the definition of the word `dyslexia' and everything it entails, there are two points all sources agree upon:
 
\begin{itemize}
\item Each individual with dyslexia is different, and is likely to present only a
	subset of the skills and deficits known to be related to dyslexia 
\item Individuals with dyslexia will most commonly have difficulty processing
	and decoding words, regardless of intelligence and cultural opportunity
\end{itemize}

Because of this, many see dyslexia as a reading disorder affecting spelling and reading acquisition. Common examples of this include confusing similarly shaped letters such as $d$, $b$, and $p$, and jumbling letters within words \cite{DetectAndManage}. However, dyslexia has also been linked to verbal memory and processing speed, affecting an individuals ability to remember verbal information such as lists and sequences, and their ability to read fluently \cite{Nhs, RoseReview}. Research in recent years has also suggested that dyslexia may manifest in areas other than phonological awareness and literacy skills \cite{snowling, DetectAndManage}.  These include confusing directions, difficulty with sequencing, and a lack of organisation skills \cite{bda}. \cite{DetectAndManage} also identifies difficulties with auditory and visual sequential memory.

The following sections consider reported symptoms of dyslexia, other than literacy problems, attempting to gain a broad knowledge about how dyslexia can really affect every area of an individuals life. The categories below in no way fully encapsulate the problems associated with dyslexia, however, provide an insight into the most commonly experienced and researched issues.

\subsection{Working Memory}
\label{sec:memory}
Individuals with dyslexia often struggle to remember list items, sequences, and even the contents of text they have just read \cite{snowling}. The reasons for this from a scientific perspective are debated, however, from a high level it seems that someone with dyslexia will put a large amount of effort into interpreting and reading the words, concentrating on understanding them as opposed to remembering them \cite{neurobiological}.
\cite{snowling} states that: ``Dyslexics typically perform poorly when their memory is assessed using tests such as the digit span task, in which sequences of digits have to be recalled in forwards and backwards order"
This suggests that the auditory memory of an individual with dyslexia can be impaired by the learning disorder, this effect was also seen by \cite{memory1980} when conducting a similar experiment. 

\subsection{Spatial Orientation}
\label{sec:spatial}
\cite{bartlett, tosee} and \cite{DetectAndManage}  all suggest that those with dyslexia are likely to have poor spatial orientation, struggling to differentiate between left and right, north, east, south and west. This is likely to make tasks such as interpreting maps and following directions difficult. 
\cite{sequential} found that some dyslexics can be distinguished from controls though the ``Block Design'' task, used as part of the WISC-R intelligence test to test spatial orientation\cite{wisc}, in which individuals must use blocks to reproduce a presented design. Individuals with dyslexia were seen to produce very good results in comparison to controls when their literacy skills were not affected by their dyslexia. Participants with dyslexia whose literacy skills \emph{were} affected were seen to score significantly worse than controls. These results suggest that this task may be able to distinguish between dyslexic and non-dyslexic individuals.

\subsection{Visual-Spatial Discrimination}
\label{sec:visualspatial}
\cite{figuresceltic} conducted a study which suggests that those with dyslexia may perform better than average, in terms of speed, when identifying impossible figures; a task designed to assess an individuals visual-spatial discrimination abilities. The study suggests that, though no more accurately, the dyslexic group tended to distinguish between impossible and possible objects significantly faster. \cite{figuresceltic} also tested participants on their ability to match complex images, through celtic matching. They found that the control group in general outperformed the dyslexic group, however, state that as the test was only using one shape their results may not be accurate. They also point out that the difference between the control and dyslexic groups was found in males only, with females in both groups performing equivalently. Despite the somewhat inconclusive results and small size of this study, the concepts may be worth testing. 

\subsection{Visual Sequential Memory}
\label{sec:visualsequentialmemory}
\cite{sequential} conducted a study across 39 dyslexic participants, with varying literacy and mathematical ability, examining their visual sequential memory. The study tested the visual sequential memory of participants by presenting them with sequences of symbols or pictures for a period of 5~seconds and then asking them to recall that sequence.  Results of this study suggested that this test is a good way be identify individuals with dyslexic from controls, with significantly better scores being obtained by controls when the objects presented were symbols.
 
\subsection{Visual Memory}
\label{sec:visualmemory}
\cite{snowlinghandbook} suggests that individuals with dyslexia are likely to have poor visual perception, including visual memory. Visual memory is described by \cite{snowlinghandbook} as:
	"The ability to remember for immediate recall all of the characteristics of a given form, and to be able to find this form from an array of similar forms"
It is suggested that weaknesses in visual perception are what cause commonly seen problems such as letter reversal and confusing letters within words, because an individual with dyslexia struggles to remember the shapes of the words. In theory this could also be applied to numbers, for example $6$ and $9$ could easily be reversed in the  same way as $b$ and $d$.

\section{Dyslexia Screening}
\label{sec:screening}
A large amount of research has been conducted into the benefits of diagnosing dyslexia early including \cite{earlyIntervention, early_computerised}, but there are always individuals who slip through the cracks. Currently it is estimated that there are over two million people in the UK alone with undiagnosed dyslexia \cite{twomillion}. Older children with dyslexia are more likely to act out behaviourally, due to lack of support, and low self-esteem \cite{behaviour}. Adults with undiagnosed dyslexia are more likely to find it more difficult to get a job, and in some cases, due to lack of specialist support when growing up, can be completely illiterate \cite{bda}. 

\cite{managing_at_uni} identifies some of the key advantages to the late diagnosis of previously undiagnosed individuals, these include the emotional benefits that come with knowing the reasons why they may have struggled at school, and no longer feeling `hopeless' or `slow'. Diagnosis is likely to limit the highlighted problems in older children and adults with dyslexia, such as behavioural difficulties, by allowing them to attain specialist help and support they would previously have been unable to access. However, before being diagnosed most individuals are required to go through a screening test which determines whether they are likely to be at risk of dyslexia and require full diagnosis. This screening test costs around \pounds 300
and is not currently administered to everybody \cite{bda}. For these reasons the target age group for the software created in this project is young adults and teens between the ages of eleven and twenty-five, at this age having left primary school, support for dyslexia begins to diminish, and individuals are less likely to be diagnosed\cite{DetectAndManage}.

\subsection{Current Screening Tests}
\label{sec:currentscreeningtests}
There are currently two standard screening tests for dyslexia in the UK which cover the target age range of this project, the \textit{Dyslexia Adult Screening Test} (DAST) and the \textit{Dyslexia  Screening Test} (DST) \cite{bda, dast}. 

The DAST is designed for adults over sixteen and the DST for children between eleven and sixteen. Both are described by \cite{dast} as screening tests, not assessments, aimed at identifying whether an individual is at risk of dyslexia, and not at dyslexia diagnosis. 

The tests include a number of tasks that the participant must complete, including a digit span task as described earlier in section \ref{sec:memory}, and rapid automatised naming (RAN).
\cite{snowling} describes RAN as a test which involves naming highly familiar objects 
under pressured conditions, these objects may include letters, digits, symbols or colours. \cite{snowling} found that:
 "Dyslexic readers take longer to complete such tasks than control children of the same age"
 
Both the DAST and DST test also include bead threading, this test notes how many beads the participant can thread within a 30 second time period, designed to examine hand eye co-ordination\cite{motorskills}.

The screening tests take a round 30~minutes to complete and also includes a number of tasks related to literacy such as testing vocabulary, letter naming, spelling, reading, and writing\cite{screeningTests}. Though not a long time individually, if administered on a large scale---for example to an entire school---the process would require weeks of a specialist psychologists time. In addition to this, the test requires participants with dyslexia to actively complete tasks that they are likely to have trouble with, such as spelling, which may cause them to become stressed and disengaged. 

\subsection{A New Screening Test?}
The DST test was first developed in 1996, closely followed by the DAST test in 1997, both over fifteen years ago \cite{dastTest, dstTest}. The creators detail in their accompanying papers very similar goals to those of this project. They created the screening test to reduce the cost and increase the accessibility of dyslexia diagnosis, allowing those who appear to have a high risk of dyslexia to be rapidly distinguished from those who do not. This meant that only individuals who were identified as high risk needed to continue with full assessment, a costly and lengthy process.

Fifteen years on and the world has changed, technology is now at the forefront of everything we do, a screening test which requires pen and paper and access to a specialist who owns the required equipment seems outdated. 
With the technology currently available more and more tasks which used to require specialised expertise are becoming available to everyone, accessible in schools, homes, and even on the go. With all this new technology being so readily available one must consider whether there is now a more cost effective and accessible way to screen for dyslexia.

\section{Games}
\label{sec:games}
Games can be physical, digital, or even mental. The definition of the word \emph{game} is therefore highly debated, with every book or paper including their own interpretation of what it means for something to be a game. \cite{halfReal} defines a game as a \emph{``rule based formal system''}, a system in which the outcome is not predetermined, stating that there must be interaction with the user in such a way that the user can influence its outcome, causing them to feel emotionally attached to that outcome. \cite{rulesOfPlay} includes nine definitions of the word game, none of which fully agree. The general consensus is that a game involves rules that enforce limitations on the users, it must be goal orientated and involve activities, events, and decision making. The definition given by \cite{artOfGameDesign} provides a nice summary of these definitions: 
\cquote{A Game is a problem-solving activity, approached with a playful
attitude}

This definition appears particularly useful as it is short and to the point, whilst encompassing a number of criteria needed in order for something to be viewed as a game. The use of \textit{problem-solving activity} suggests that there must be some form of challenge within the game, an obstacle for the user to overcome. Games, under this definition, must be interactive, allowing the user to participate and control the flow of play. This means there must be rules governing the bounds of what the user can and cannot do. A \textit{playful attitude} suggests that the user should want to play the game, it should be a fun and interesting experience whilst providing the necessary challenge.

Though the definition provided by \cite{artOfGameDesign} in no way fully expresses what a game is or can be, it appears the most suitable for how the word \emph{game} should be interpreted in the context of this paper, which focusses on digital games. Particular emphasis is placed upon the need for a game to be a playful experience, the user should want to play the game and be fully immersed within its story. Users should not feel like they are being tested or forced to play against their will.

\subsection{Serious Digital Games}
Serious games are a subset of games that aim to achieve something more than just user experience and entertainment\cite{stegeserious}. They are used in many fields, including politics, health, and education, and have serious goals such as skill acquisition, or identifying users with a particular trait, skill or deficit\cite{SeriousOverview}.
 
\cite{SeriousOverview} identifies the differences between serious games and those designed purely for entertainment, determining that serious games are more likely to value achieving their serious goal above rich user experience and learning above fun. This creates the question:
\cquestion{What if a rich user experience is required in order to solve the serious problem}

For example, when attempting to realistically simulate an environment in order to test a skill, the user must be completely immersed in the game environment in order to perform as they would when confronted with the same situation in reality. 
In the case of the work conducted in this paper, solving the problem and the user experience are equally important. The serious goal of this work is to accurately map users to one of two groups, those with dyslexia and those without, this accuracy is only achievable if players fully engage with the process, making user experience just as important. 

The decision to make user experience equally as important as achieving the serious goal creates another goal: engaging the user. When defining the word game, it was strongly emphasised that the user should enter the game willingly, and not feel like they are being tested or forced to play, however, the user will be being tested and, in some cases, will be forced to play by a parent or teacher, this creates another question:
\cquestion{How can a game be created which hides this from the user}

The answer appears to be to carefully examine the mechanics, aesthetics, story and technology of the game in order to design a game which successfully abstracts its serious nature and goals from the user.

\subsection{Serious Games for Health and Education}
In recent years games have become ever more common as a method of education\cite{stapleton2004}, allowing players to learn whilst being entertained.
 \cite{stapleton2004} describes how people who play games can be seen as learners, with different players learning the concepts and goals of a game at different rates. \cite{stapleton2004} compares learning through games and learning through school, stating that games encourage players to learn as it is their choice, unlike in a school environment where they are told what they must learn and when. This suggests that people may learn better through games.
The game \emph{Supercharged} is given as an example of a serious game designed for learning, specifically teaching players about physics\cite{supercharged}. Results of an experiment involving \emph{Supercharged} found that the game helped children to understand basic physics concepts, however, often children---particularly boys---became disengaged with the game after a short time\cite{supercharged}, likely due to its repetitive nature.

Games have also been used as teaching tools for those with disabilities such as autism spectrum disorder. A nice example of the use of games to aid individuals with autism is the \emph{Let's Face It} program, aimed at teaching children face recognition skills\cite{faceit}. \cite{faceit} accepts that the games within the \emph{Let's Face It} program are limited, however, they still manage to show promising results and appear particularly excited at the idea of producing a free tool available to everyone at home, school, or elsewhere. 

Games to help with dementia have also been investigated,
\cite{dementia} describes games and puzzles that have been created to help tackle the memory, visual attention and coordination problems associated with dementia, including digital mazes and jigsaws.

The above show that recently serious digital games have been used in an attempt to solve existing and prominent issues, both in the health and education sectors.

\subsection{Serious Games for Dyslexia}
Previously, Section \ref{sec:dyslexia} discussed common skills and deficits associated with dyslexia, with a focus on tasks which involve minimal reading, writing and spelling. This has been done in order to allow a dyslexia screening game to be created which is as appealing and accessible as possible to those who struggle with these vital skills. It is unlikely that individuals with these difficulties will engage with a game which so openly points out their known weaknesses, instead the game content must be designed such that the link with dyslexia is as subtle as possible. In addition to this, because the game will not rely  heavily upon the players English language skills, it may easily port to other languages.

Accessibility was a key factor in choosing to experiment with using a game as a dyslexia screening test. Most individuals understand the concept of a game from a very early age, and in todays society games are commonly digital, played on a computer, console or mobile device. %%CITE
Having a game as a dyslexia screening test takes advantage of the users existing knowledge and familiarity with games, making them feel at ease when playing, and making the experience fun and engaging. If the user feels connected to the game, they will be more likely to perform to their best ability making the test more accurate. In addition to this, a digital game has relatively low cost, with no specialist equipment 
required other than a device with which to play, making it much more accessible than the current screening tests. On top of this, most households in the UK have access to a mobile device or computer already, meaning the extra cost involved is nil\cite{2013GamesData}. 
The availability of digital devices means the catchment of the game could be enormous, an entire school could be tested, using existing resources, within a few hours with no additional cost other than electricity. This is a stark contrast to the specialist teachers and equipment currently required.

There has been previous attempts to complete similar work, however, very little in the way of evaluation of the games performance as a screening test.

\cite{SeriousForPredicting} uses the serious game paradigm in an attempt to maintain user attention, and encourage full participation in their study, much like the reasoning behind using games within this project. They recognise that participants are more likely to perform the required exercises if the tasks do not appear to be boring, and they do not feel like they are being tested. The paper describes the creation of a set of serious games, targeted at predicting the risk of dyslexia in pre-readers. The target audience of the games demand that the tasks completed by participants are primarily non verbal, though an alphabetic theme runs throughout.  Despite the papers title: \emph{``A serious game for predicting the risk of dyslexia in pre-readers''}, it does not describe how a serious game can be used to predict the risk of dyslexia, but instead concentrates sole on user experience.

\cite{Dyseggxia} is a paper which describes the creation of \textit{Dyseggxia}, a serious game designed to train the literacy skills of dyslexic individuals and not as a tool for screening or diagnosis.  Unlike this project, \cite{Dyseggxia} makes little attempt to hide its serious goals from users, nor does it attempt to limit the users exposure to tasks they are likely to find difficult, making them repeat the same task over and over until they work out the correct answer. Since the game is designed to train literacy skills it is inevitable that they will be at the forefront of the game, however, this does not make for an engaging experience. It is very obvious this game is designed for learning and little else. 

\section{Game Design}
\label{sec:gamedesign}

Following the guidance provided by \cite{artOfGameDesign} the genre, theme, and mechanics of the dyslexia screening game were designed. The game was chosen to be of the casual \emph{2D Platform} genre, allowing to players to control a characters movement in the x-axis, jump, and interact with game objects. The simple nature and popularity of this genre will hopefully mean that there will be no bias towards users with a strong gaming background, with little for new users to learn. 

There seems not to be a tried and tested method of deciding upon the game theme, instead the game designers creativity is tested. With this in mind the chosen theme for the screening game is:
\begin{center}
\textit{\textbf{The fantasy of being a lost robot}}
\end{center}
The user plays the game as a lost and lonely robot, attempting to escape an underground sewer. The robot must navigate across platforms  through the sewer, utilising objects and completing puzzles as they go.

Having decided upon the games genre and theme the research about dyslexia must be incorporated, the most suitable way to integrate this research into the game environment was through screening tasks. 

\subsection{Screening Tasks}
\label{sec:screening_Tasks}
A Screening task within this context can be seen as a puzzle or sub-game within the main game. They are specifically designed to highlight issues identified as being related to dyslexia. 
For this initial version of the game five screening tasks have been designed, as explained below.

\subsubsection{Working Memory} 
Working memory is tested by presenting the user with visual sequences of varying lengths, the user must remember and repeat the sequence. It is expected, based on the research, that individuals with dyslexia are likely to perform worse on this task, taking longer and making more mistakes, for this reason time taken and number of errors made will be collected as metrics for this task.

\subsubsection{Spatial Orientation} 
Based not the block design task of the WISC-R intelligence test, in the spatial orientation screening task the user is presented with a blueprint of a circuit design and must rotate the components of the actual circuit to match. Research suggests that, depending on the type of dyslexia an individual has, those with dyslexia are likely to sit at either end of the spectrum outperforming and under performing controls. Time taken will be collected for this screening task.

\subsubsection{Visual-Spatial Discrimination} 
Emulating the celtic matching task, previously discussed, this screening task presents the user with a complex lock and asks them to select the key that matches from a set of similar looking keys. The research surrounding this task is somewhat inconclusive s to expected results, however, both time taken and number of errors made will be kept as metrics for this task.


\section{Evaluation}
\subsection{Method}
\subsubsection{Design}
\subsubsection{Apparatus and Materials}
\subsubsection{Participants}
\subsubsection{Procedure}
\subsection{Results}
\subsection{Discussion}
\section{Conclusions}
\subsection{Further Work}
\bibliography{references}
\bibliographystyle{IEEEtran}
\section{Appendix}
\end{document}


