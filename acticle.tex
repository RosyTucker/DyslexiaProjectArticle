\documentclass[journal]{IEEEtran}
\usepackage[pdftex]{graphicx}
\usepackage[table]{xcolor}
\usepackage{parskip}
\usepackage{float}
\usepackage[latin1]{inputenc}
\usepackage{tikz}
\usetikzlibrary{shapes,arrows}

\begin{document}
\title{\textbf{A Serious Game for Aiding the Screening of Dyslexia in Children and Young Adults}}
\author{Rose Tucker}

\maketitle


\begin{abstract}
The standard screening tests for detecting dyslexia in adolescents typically involve a number of written,
 spoken, and visual tests to be carried out by a specialist dyslexic teacher or psychologist.
This test lasts around half an hour and can cost the individual in excess of \textsterling300. 
This paper presents a more accessible and engaging method of screening individuals
 for dyslexia, through the use of a serious game. 
The game was tested on 42 participants, 7 diagnosed with dyslexia, and 35 controls which were seen to be in no way dyslexic. 
Testing showed that two gameplay performance metrics identify significant differences between the two groups, and that the game can identify diagnosed dyslexic subjects from those without dyslexia with an accuracy of 95.2\%, whilst providing an engaging experience for all users.
\end{abstract}

\section{ Section}
\IEEEPARstart{S}{action} content

\subsection{Subsection}
Subsection Content
\bibliography{references}
\bibliographystyle{IEEEtran}
\section{Appendix}
\end{document}


