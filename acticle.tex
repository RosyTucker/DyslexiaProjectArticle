\documentclass[journal]{IEEEtran}
\usepackage[pdftex]{graphicx}
\usepackage[table]{xcolor}
\usepackage{parskip}
\usepackage{float}
\usepackage{tikz}
\usetikzlibrary{shapes,arrows}

\begin{document}
\title{\textbf{A Serious Game for Aiding the Screening of Dyslexia in Children and Young Adults}}
\author{Rose Tucker}

\maketitle


\begin{abstract}
The standard screening tests for detecting dyslexia in adolescents typically involve a number of written,
 spoken, and visual tests to be carried out by a specialist dyslexic teacher or psychologist.
This test lasts around half an hour and can cost the individual in excess of \textsterling300. 
This paper presents a more accessible and engaging method of screening individuals
 for dyslexia, through the use of a serious game. 
The game was tested on 42 participants, 7 diagnosed with dyslexia, and 35 controls which were seen to be in no way dyslexic. 
Testing showed that two gameplay performance metrics identify significant differences between the two groups, and that the game can identify diagnosed dyslexic subjects from those without dyslexia with an accuracy of 95.2\%, whilst providing an engaging experience for all users.
\end{abstract}

\section{Introduction}
\IEEEPARstart{T}{argeting} adolescents, this work examines whether the conventional screening
test for dyslexia could be replaced with a serious game. The game aims to use gameplay performance metrics of a player to predict wether they are likely to have dyslexia. The game also aims to screen individuals without explicitly highlighting the literacy and phonological deficits often associated with dyslexia, with the hope of 
producing an engaging experience for those with and without dyslexia. 
If the game is successful, it should vastly reduce the cost, personnel, and time taken to 
identify dyslexia through not requiring a specialist dyslexia teacher, or psychologist, to 
conduct a screening test \cite{bda, dast}. Hopefully the accessibility of the game, to 
teachers, parents, and students themselves, will result in more
young people being tested and receiving the specialist help they require.

The following sections examine the current literature surrounding both dyslexia and games, to determine the breadth of current knowledge in both areas and show how combining the two fields could change what is state-of-the-art. 

\section{Dyslexia}
\label{sec:dyslexia}

Dyslexia is a specific learning disability which affects around 8-10\% of the UK population \cite{Nhs,bda}. It is often considered a continuum, with no distinct cut off and varying severity. Though there are disagreements as to the definition of the word `dyslexia' and everything it entails, there are two points all sources agree upon:
 
\begin{itemize}
\item Each individual with dyslexia is different, and is likely to present only a
	subset of the skills and deficits known to be related to dyslexia 
\item Individuals with dyslexia will most commonly have difficulty processing
	and decoding words, regardless of intelligence and cultural opportunity
\end{itemize}

Because of this, many see dyslexia as a reading disorder affecting spelling and reading acquisition. Common examples of this include confusing similarly shaped letters such as $d$, $b$, and $p$, and jumbling letters within words \cite{DetectAndManage}. However, dyslexia has also been linked to verbal memory and processing speed, affecting an individuals ability to remember verbal information such as lists and sequences, and their ability to read fluently \cite{Nhs, RoseReview}. Research in recent years has also suggested that dyslexia may manifest in areas other than phonological awareness and literacy skills \cite{snowling, DetectAndManage}.  These include confusing directions, difficulty with sequencing, and a lack of organisation skills \cite{bda}. \cite{DetectAndManage} also identifies difficulties with auditory and visual sequential memory.

The following sections consider reported symptoms of dyslexia, other than literacy problems, attempting to gain a broad knowledge about how dyslexia can really affect every area of an individuals life. The categories below in no way fully encapsulate the problems associated with dyslexia, however, provide an insight into the most commonly experienced and researched issues.

\subsubsection{Working Memory}

\subsubsection{Spatial Orientation}

\subsubsection{Visual-spatial Discrimination}

\subsubsection{Visual Sequential Memory}

\subsubsection{Visual Memory}


\bibliography{references}
\bibliographystyle{IEEEtran}
\section{Appendix}
\end{document}


