\documentclass[journal]{IEEEtran}
\usepackage[pdftex]{graphicx}
\usepackage[table]{xcolor}
\usepackage{parskip}
\usepackage{float}
\usepackage{tikz}
\usetikzlibrary{shapes,arrows}

\begin{document}
\title{\textbf{A Serious Game for Aiding the Screening of Dyslexia in Children and Young Adults}}
\author{Rose Tucker}

\maketitle


\begin{abstract}
The standard screening tests for detecting dyslexia in adolescents typically involve a number of written,
 spoken, and visual tests to be carried out by a specialist dyslexic teacher or psychologist.
This test lasts around half an hour and can cost the individual in excess of \textsterling300. 
This paper presents a more accessible and engaging method of screening individuals
 for dyslexia, through the use of a serious game. 
The game was tested on 42 participants, 7 diagnosed with dyslexia, and 35 controls which were seen to be in no way dyslexic. 
Testing showed that two gameplay performance metrics identify significant differences between the two groups, and that the game can identify diagnosed dyslexic subjects from those without dyslexia with an accuracy of 95.2\%, whilst providing an engaging experience for all users.
\end{abstract}

\section{Introduction}
\IEEEPARstart{T}{argeting} adolescents, this work examines whether the conventional screening
test for dyslexia could be replaced with a serious game. The game aims to use gameplay performance metrics of a player to predict wether they are likely to have dyslexia. The game also aims to screen individuals without explicitly highlighting the literacy and phonological deficits often associated with dyslexia, with the hope of 
producing an engaging experience for those with and without dyslexia. 
If the game is successful, it should vastly reduce the cost, personnel, and time taken to 
identify dyslexia through not requiring a specialist dyslexia teacher, or psychologist, to 
conduct a screening test\cite{bda, dast}. Hopefully the accessibility of the game, to 
teachers, parents, and students themselves, will result in more
young people being tested and receiving the specialist help they require.

The following sections examine the current literature surrounding both dyslexia and games, to determine the breadth of current knowledge in both areas and show how combining the two fields could change what is state-of-the-art. 

\bibliography{references}
\bibliographystyle{IEEEtran}
\section{Appendix}
\end{document}


